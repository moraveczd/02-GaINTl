\documentclass[hyperref=unicode,presentation,10pt]{beamer}

\usepackage[absolute,overlay]{textpos}
\usepackage{array}
\usepackage{graphicx}
\usepackage{adjustbox}
\usepackage[version=4]{mhchem}
\usepackage{chemfig}
\usepackage{caption}

%dělení slov
\usepackage{ragged2e}
\let\raggedright=\RaggedRight
%konec dělení slov

\addtobeamertemplate{frametitle}{
	\let\insertframetitle\insertsectionhead}{}
\addtobeamertemplate{frametitle}{
	\let\insertframesubtitle\insertsubsectionhead}{}

\makeatletter
\CheckCommand*\beamer@checkframetitle{\@ifnextchar\bgroup\beamer@inlineframetitle{}}
\renewcommand*\beamer@checkframetitle{\global\let\beamer@frametitle\relax\@ifnextchar\bgroup\beamer@inlineframetitle{}}
\makeatother
\setbeamercolor{section in toc}{fg=red}
\setbeamertemplate{section in toc shaded}[default][100]

\usepackage{fontspec}
\usepackage{unicode-math}

\usepackage{polyglossia}
\setdefaultlanguage{czech}

\def\uv#1{„#1“}

\mode<presentation>{\usetheme{default}}
\usecolortheme{crane}

\setbeamertemplate{footline}[frame number]

\title[Crisis]
{C2062 -- Anorganická chemie II}

\subtitle{(Bor, hliník,) gallium, indium, thallium a nihonium}
\author{Zdeněk Moravec, hugo@chemi.muni.cz \\ \adjincludegraphics[height=60mm]{img/IUPAC_PSP.jpg}}
\date{}

\begin{document}

\begin{frame}
	\titlepage
\end{frame}

\section{Bor}
\frame{
	\frametitle{}
	\begin{itemize}
		\item Bor je nekovový prvek, značka \textbf{B}, protonové číslo 5.
		\item Ve vesmíru je velmi vzácný, koncentrace na Zemi se pohybuje okolo 9~ppm.
		\item Borax -- \ce{Na2[B4O5(OH)4].8H2O} -- byl znám už od starověku.
		\item Čistý bor byl připraven až ve 20. století.
		\item Přírodní bor se skládá ze dvou izotopů, $^{10}$B a $^{11}$B. Jejich poměr se liší podle lokality, což znemožňuje přesné stanovení atomové hmotnosti.
		\item Jádro $^{11}$B (I = 3) se využívá v NMR spektroskopii.\footnote[frame]{\href{https://doi.org/10.1021/cr00010a007}{Boron-11 NMR spectra of boranes, main-group heteroboranes, and substituted derivatives}}
		\item Známe čtyři základní metody přípravy boru:
		\begin{itemize}
			\item Redukcí kovy za vysokých teplot:
			\item \ce{2 BCl3 + 3 Zn ->[900 $^\circ$C] 2 B + 3 ZnCl2}
			\item Elektrolýzou boritanů nebo tetrafluoboritanů (\ce{KBF4}).
			\item Redukcí těkavých sloučenin boru (\ce{BBr3}) na žhaveném tantalovém vlákně (900--1500~$^\circ$C). Tato metoda poskytuje nejčistší bor ve vysokých množstvích.
			\item Termický rozklad halogenidů a hydridů boru.
		\end{itemize}
	\end{itemize}
}

\frame{
	\frametitle{}
	\begin{itemize}
		\item Elementární bor vytváří velké množství allotropních modifikací.
		\item Bor má k dispozici pouze tři valenční elektrony a má příliš malý iontový poloměr, takže nemůže vytvářet kovovou vazbu. Proto jsou velmi zajímavé vazebné poměry ve sloučeninách boru i v jeho allotropech.
		\item Jejich základní strukturní jednotkou je zpravidla ikosaedr \ce{B12} s~pětičetnou rotační symetrií.
		\item Díky své symetrii se ikosaedry seskládají s poměrně velkými dutinami, které umožňují umístění dalších atomů boru, příp. jiných prvků.
		\item Struktura $\alpha$-boru se skládá z ikosaedrů \ce{B12} v nejtěsnějším hexagonálním uspořádání.
		\item Termodynamicky nejstabilnější formou je $\beta$-bor, jeho základní buňka obsahuje 105 atomů boru, středový ikosaedr je obklopen dalšími ikosaedry, které jsou do něj vnořeny.
		\item Struktura $\gamma$-boru odpovídá strukturnímu typu NaCl, střídají se zde ikosaedry \ce{B12} a dvojice \ce{B2}.
	\end{itemize}
}

\frame{
	\frametitle{}
	\begin{columns}
		\begin{column}{.5\textwidth}
			\begin{figure}
				\adjincludegraphics[height=.5\textheight]{img/Alpha_AlB12_Ikosaeder.png}
				\caption*{Ikosaedr \ce{B12}.\footnote[frame]{\href{https://commons.wikimedia.org/wiki/File:Alpha_AlB12_Ikosaeder.png}{Zdroj: Andif1/Commons}}}
			\end{figure}
		\end{column}

		\begin{column}{.5\textwidth}
			\begin{figure}
				\adjincludegraphics[width=.8\textwidth]{img/Betaboron.jpg}
				\caption*{Základní buňka $\beta$-boru.\footnote[frame]{\href{https://commons.wikimedia.org/wiki/File:Betaboron.jpg}{Zdroj: Materialscientist/Commons}}}
			\end{figure}
		\end{column}
	\end{columns}
}

\frame{
	\frametitle{}
	\begin{figure}
		\adjincludegraphics[height=.7\textheight]{img/Gamma-bor.jpg}
		\caption*{Krystalová struktura $\gamma$-boru.\footnote[frame]{\href{https://commons.wikimedia.org/wiki/File:Gamma-bor.jpg}{Zdroj: Materialscientist/Commons}}}
	\end{figure}
}

\frame{
	\frametitle{}
	\begin{itemize}
		\item Vzhledem k malému počtu valenčních elektronů musí bor využívat exotičtější vazebné možnosti.
		\item V roce 1949 byla zavedena představa třístředové dvouelektronové vazby v boranech (hydridech boru).
		\item Podle teorie LCAO-MO vznikají molekulové orbitaly (MO) kombinací dvou atomových orbitalů. Tímto způsobem vznikne vazebný a protivazebný MO.
		\item Kombinovat se ale může více AO za vzniku stejného počtu MO (vazebných, nevazebných a protivazebných).
	\end{itemize}
	\begin{figure}
		\adjincludegraphics[height=.35\textheight]{img/Diborane_02.png}
	\end{figure}
}

\frame{
	\frametitle{}
	\begin{itemize}
		\item V chemii boranů se setkáváme s dvěma typy třístředových dvouelektronových vazeb:
		\begin{itemize}
			\item \ce{B-H-B}
			\item \ce{B-B-B}
		\end{itemize}
	\end{itemize}
	\begin{columns}
		\begin{column}{.5\textwidth}
			\begin{figure}
				\adjincludegraphics[height=.5\textheight]{img/Diborane_B-H-B_MO_diagram.jpg}
				\caption*{Molekulové orbitaly \ce{B-H-B}\footnote[frame]{\href{https://commons.wikimedia.org/wiki/File:Diborane_B-H-B_MO_diagram.pdf}{Zdroj: Albris/Commons}}}
			\end{figure}
		\end{column}

		\begin{column}{.5\textwidth}
			\begin{figure}
				\adjincludegraphics[height=.35\textheight]{img/Diborane_02.png}
			\end{figure}
		\end{column}
	\end{columns}
}

\frame{
	\frametitle{}
	\begin{tabular}{|l|c|}
		\hline
		Diboran(2) & \ce{HB=BH} \\\hline
		\begin{tabular}{l}
			Diboran(4)\footnote[frame]{\href{https://commons.wikimedia.org/wiki/File:Diborane4.png}{Zdroj: DMacks/Commons}}
		\end{tabular} & \adjincludegraphics[height=.3\textheight]{img/Diborane4.png} \\\hline
		Diboran(6)\footnote[frame]{\href{https://commons.wikimedia.org/wiki/File:Diborane-2D.png}{Zdroj: Benjah-bmm27/Commons}} & \adjincludegraphics[height=.3\textheight]{img/Diborane-2D.png}\\
		\hline
	\end{tabular}
}

\section{Hliník}
\frame{
	\frametitle{}
	\begin{itemize}
		\item Hliník je kovový prvek, značka \textbf{Al}, protonové číslo 13.
		\item Hliník je nejrozšířenějším kovem v zemské kůře, jeho koncentrace je 8,1 \%.\footnote[frame]{\href{https://www.worldatlas.com/articles/the-most-abundant-elements-in-the-earth-s-crust.html}{The Most Abundant Elements In The Earth's Crust}}
		\item Jádro $^{27}$Al (I = $\frac{5}{2}$) se využívá v NMR spektroskopii.\footnote[frame]{\href{https://doi.org/10.1016/j.pnmrs.2016.01.003}{Recent advances in application of $^{27}$Al NMR spectroscopy to materials science}}
		\item Je součástí hlinitokřemičitanů jako jsou jíly, slídy a živce, hlavním minerálem, resp. směsí minerálů je bauxit (hydratovaný oxid hlinitý, \ce{Al2O3.2H2O}). Skládá se z hydroxidů hlinitých, gibbsitu, böhmitu a~dalších.
		\item Je kujný a velmi dobře vede elektrický proud.
	\end{itemize}
	\begin{figure}
		\adjincludegraphics[height=.25\textheight]{img/Lingot_aluminium.jpg}
		\caption*{Hliníkový ingot.\footnote[frame]{\href{https://commons.wikimedia.org/wiki/File:Lingot_aluminium.jpg}{Zdroj: Romary/Commons}}}
	\end{figure}
}

\frame{
	\frametitle{}
	\begin{itemize}
		\item V kovovém stavu je velmi reaktivní, na vzduchu se pasivuje tvorbou povrchové, kompaktní vrstvy \ce{Al2O3}.
		\item Vytváří sloučeniny v oxidačních číslech +I až +III, nejstabilnější a nejběžnější jsou hlinité sloučeniny.
		\item Je amfoterní, rozpouští se ve zředěných kyselinách i hydroxidech za vývoje vodíku:
		\item \ce{2 Al + 2 NaOH + 6 H2O -> 2 Na[Al(OH)4] + 3 H2}
		\item \ce{2 Al + 3 H2SO4 -> Al2(SO4)3 + 3 H2}
		\item Jemný, práškový hliník reaguje explozivně s kapalným kyslíkem.\footnote[frame]{\href{https://doi.org/10.1021/ed036p54}{Explosive hazard of aluminum-liquid oxygen mixtures}}
	\end{itemize}
}

\frame{
	\frametitle{}
	\begin{itemize}
		\item Vysoké afinity hliníku ke kyslíku se využívá v aluminotermii.\footnote[frame]{\href{http://test.sciencezoom.cz/apps/zf_08/?target=efektni_3&pokus=pokus_1}{Aluminotermie}}
		\item Směs \ce{Al + Fe2O3} se označuje jako \textit{termit} a dříve se využívala ke svařování.\footnote[frame]{\href{https://www.youtube.com/watch?v=ENvjFV8TXmU}{Svařování termitem na nádraží v Židlochovicích}}
		\item \ce{2 Al + Fe2O3 -> Al2O3 + 2 Fe}
	\end{itemize}
	\begin{figure}
		\adjincludegraphics[height=.4\textheight]{img/Aluminothermic_reduction.jpg}
		\caption*{Aluminotermická reakce.\footnote[frame]{\href{https://commons.wikimedia.org/wiki/File:Aluminothermic_reduction_of_chromium_oxide_(2).JPG}{Zdroj: Rando Tuvikene/Commons}}}
	\end{figure}
}

\frame{
	\frametitle{}
	\begin{itemize}
		\item Hlavní rudou je bauxit, \ce{Al2O3.2 H2O}.
		\item Surový bauxit obsahuje velké množství nečistot, převážně \ce{Fe2O3}, \ce{SiO2} a \ce{TiO2}.\footnote[frame]{\href{https://www.youtube.com/watch?v=1u8gzoT-seg}{How It's Made - Aluminium or Aluminum}}
		\item K čištění se využívá nejčastěji \textit{Bayerův způsob}:\footnote[frame]{\href{https://oenergetice.cz/elektrina/zpracovani-hliniku-od-mineralu-az-po-kabel}{Zpracování hliníku - od minerálu až po kabel}}
		\begin{itemize}
			\item Bauxit se rozdrtí a rozemele.
			\item Pod tlakem, za teploty 180~$^\circ$C se k němu přidá NaOH.
			\item Odfiltrují se nerozpuštěný zbytek a vzniklý roztok \ce{Na[Al(OH)4]} je ochlazen a očkován pomocí krystalů \ce{Al(OH)3}. Nebo je sycen oxidem uhličitým, který vysráží hliník v podobě \ce{\alpha-Al(OH)3}.
			\item Bezvodý \ce{Al2O3} se získá tepelným rozkladem \ce{Al(OH)3}.
		\end{itemize}
		\item Hliník se z oxidu hlinitého získává elektrolyticky, \textit{Hallovým-Heroultovým procesem}.\footnote[frame]{\href{https://www.acs.org/content/acs/en/education/whatischemistry/landmarks/aluminumprocess.html}{Production of Aluminum: The Hall-Héroult Process}}
		\item Jako tavidlo se využívá kryolit (\ce{Na3AlF6}), čímž dojde k výraznému snížení teploty tání.
		\item Anoda je uhlíková, katodu tvoří tavenina hliníku.
		\item \ce{2 Al2O3 + 3 C ->[960 $^\circ$C] 4 Al + 3 CO2}
	\end{itemize}
}

\frame{
	\frametitle{}
	\begin{figure}
		\adjincludegraphics[height=.7\textheight]{img/Hall-heroult.png}
		\caption*{Řez elektrolyzérem.\footnote[frame]{Zdroj: \href{https://commons.wikimedia.org/wiki/File:Hall-heroult-kk-2008-12-31.png}{Kashkan/Commons}}}
	\end{figure}
}

\frame{
	\frametitle{}
	\vfill
	\begin{itemize}
		\item Kovový hliník se využívá jako konstrukční materiál pro automobilový a letecký průmysl, obalový materiál a jako elektrický materiál (kabeláž, motory, transformátory).\footnote[frame]{\href{https://www.aluminiumleader.com/application/electrical_engineering/}{Aluminum in Power-Engineering}}
		\item Tenká hliníková fólie se využívá jako záznamové médium v CD-ROM discích. Data jsou uložena ve formě vrypů a čtení se realizuje pomocí odrazu LASERu od povrchu fólie.\footnote[frame]{\href{https://pctechnicalpro.blogspot.com/2017/04/how-does-cd-rom-works-working-principle.html}{How Does A CD-ROM Works? Working Principle Of CD-ROM}}
		\item Velice důležité jsou oxidy hliníku, korund (\ce{\alpha-Al2O3}) a smirek (korund, magnetit a hematit). Mají vysokou tvrdost a využívají se jako brusné materiály.
		\item Korund se také používá pro výrobu kelímků pro vysokoteplotní aplikace.
		\item Sloučeniny hliníku jsou důležité v katalýze, např. Friedel-Crafts.\footnote[frame]{\href{https://www.masterorganicchemistry.com/2011/07/22/reagent-friday-aluminum-chloride-alcl3/}{Reagent Friday: Aluminum Chloride (\ce{AlCl3})}}
	\end{itemize}
	\vfill
}

\frame{
	\frametitle{}
	\vfill
	\begin{columns}
		\begin{column}{.5\textwidth}
			\begin{figure}
				\adjincludegraphics[height=.55\textheight]{img/Cans_recycling.jpg}
				\caption*{Hliníkové plechovky na recyklaci.\footnote[frame]{Zdroj: \href{https://commons.wikimedia.org/wiki/File:Cans_recycling.jpg}{Radulf del Maresme/Commons}}}
			\end{figure}
		\end{column}

		\begin{column}{.5\textwidth}
			\begin{figure}
				\adjincludegraphics[height=.55\textheight]{img/DTA-_und_DSC-Tiegel.jpg}
				\caption*{Korundový kelímek pro DTA (vlevo).\footnote[frame]{Zdroj: \href{https://commons.wikimedia.org/wiki/File:DTA-_und_DSC-Tiegel.jpg}{Bic/Commons}}}
			\end{figure}
		\end{column}
	\end{columns}
	\vfill
}

\frame{
	\frametitle{}
	\vfill
	\begin{itemize}
		\item Hliník, na rozdíl od boru, nevyužívá elektron-deficitní vazby. Ve sloučeninách má zpravidla koordinační číslo v rozsahu 4--6.
		\item Například \textit{chlorid hlinitý}, \ce{AlCl3},tvoří v pevném stavu polymerní krystaly. Při tání se pak rozpadá na dimerní molekuly \ce{Al2Cl6}.
		\item Další zahřívání vede k rozpadu na trigonálně planární molekuly \ce{AlCl3}.
		\item Vytváří hexahydrát, \ce{[Al(H2O)6]Cl3}.
	\end{itemize}
	\begin{columns}
		\begin{column}{.35\textwidth}
		\begin{figure}
		\adjincludegraphics[width=\textwidth]{img/Aluminium-trichloride-dimer-3D-balls.png}
		\caption*{Struktura dimeru.\footnote[frame]{\href{https://commons.wikimedia.org/wiki/File:Aluminium-trichloride-dimer-3D-balls.png}{Zdroj: Benjah-bmm27/Commons}}}
		\end{figure}
		\end{column}
		\begin{column}{.65\textwidth}
		\begin{figure}
			\adjincludegraphics[width=.73\textwidth]{img/Aluminium-chloride-xtal.png}
			\caption*{Krystalová struktura \ce{AlCl3}.\footnote[frame]{\href{https://commons.wikimedia.org/wiki/File:Aluminium-chloride-xtal-viewed-down-c-axis-3D-bs-17.png}{Zdroj: Ben Mills/Commons}}}
		\end{figure}
		\end{column}
	\end{columns}
	\vfill
}

\section{Úvod -- gallium, indium, thallium a nihonium}
\frame{
	\frametitle{}
	\vfill
	\begin{tabular}{|c|l|l|l|}
	\hline
	 & \textit{Gallium} & \textit{Indium} & \textit{Thallium} \\\hline
	 El. konfigurace & 3d$^{10}$ 4s$^{2}$ 4p$^{1}$ & 4d$^{10}$ 5s$^{2}$ 5p$^{1}$ & 4f$^{14}$ 5d$^{10}$ 6s$^{2}$ 6p$^{1}$ \\\hline
	 Teplota tání [$^\circ$C] & 29,76 & 156,60 & 304 \\\hline
	 Teplota varu [$^\circ$C]  & 2400 & 2072 & 1473 \\\hline
	 Objeven & 1875 & 1863 & 1861 \\\hline
	 Vzhled & modrostříbrný\footnote[frame]{\href{https://commons.wikimedia.org/wiki/File:Gallium_crystals.jpg}{Zdroj: Foobar/Commons}} & stříbrný\footnote[frame]{\href{https://commons.wikimedia.org/wiki/File:Indium.jpg}{Zdroj: Nerdtalker/Commons}} & stříbrobílý\footnote[frame]{\href{https://commons.wikimedia.org/wiki/File:Thallium_pieces_in_ampoule.jpg}{Zdroj: W. Oelen/Commons}} \\
	 &  \begin{minipage}{.2\textwidth}
	 	\adjincludegraphics[width=\linewidth]{img/Gallium.jpg}
	 \end{minipage}
	 	& \begin{minipage}{.2\textwidth}
	 		\adjincludegraphics[width=\linewidth]{img/Indium.jpg}
	 	\end{minipage} & \begin{minipage}{.2\textwidth}
	 	\adjincludegraphics[width=\linewidth]{img/Thallium.jpg}
 	\end{minipage} \\\hline
	\end{tabular}
	\vfill
}

\frame{
	\frametitle{}
	\vfill
	\begin{itemize}
		\item \textit{Nihonium}
		\begin{itemize}
			\item Umělý prvek, protonové číslo 113, Nh.
			\item Poprvé byl připraven v roce 2003:
			\item \ce{^{243}_{95}Am + ^{48}_{20}Ca -> ^{291}_{115}Mc$^\star$ -> ^{288}_{115}Mc + 3 n -> ^{284}Nh + $\alpha$}
			\item \ce{^{243}_{95}Am + ^{48}_{20}Ca -> ^{291}_{115}Mc$^\star$ -> ^{287}_{115}Mc + 4 n -> ^{283}Nh + $\alpha$}
			\item T$_\frac{1}{2}$ ($^{268}$Nh) = 20 s
			\item \ce{^{286}_{113}Nh -> ^{282}_{111}Rg + $\alpha$}
		\end{itemize}
		\item Pojmenován byl po Japonsku -- „země vycházejícího slunce“.
		\item Známe osm izotopů $^{278}$Nh -- $^{290}$Nh, nejdelší poločas rozpadu má $^{286}$Nh, $t_\frac{1}{2}$ = 9,5~s.
		\item Chemické vlastnosti nihonia nebyly zatím detailně prozkoumány.\footnote[frame]{\href{https://arxiv.org/abs/1212.4292}{First foot prints of chemistry on the shore of the Island of Superheavy Elements}}
	\end{itemize}
	\vfill
}

\section{Chemické a fyzikální vlastnosti}
\frame{
	\frametitle{}
	\vfill
	\begin{itemize}
		\item Gallium a indium nejsou toxické, ale thallium se řadí mezi extrémně toxické prvky.\footnote[frame]{\href{https://www.webelements.com/thallium/}{Thallium: the essentials}}
		\item Gallium krystaluje v orthorombické krystalové soustavě.
		\item Indium má tetragonální plošně centrovanou strukturu, každý atom india sousedí se čtyřmi dalšími ve vzdálenosti 324~pm a osmi ve vzdálenosti 336~pm.
		\item Thallium krystaluje v nejtěsnějším hexagonálním uspořádání (HCP).
		\item Teploty tání jsou poměrně nízké.
		\item Všechny tři kovy vytvářejí sloučeniny v oxidačních číslech I a III.
		\item Stálost oxidačního stavu I stoupá v řadě Al $<$ Ga $<$ In $<$ Tl.
		\item Oxidy a hydroxidy kovů v oxidačním stavu I jsou zásaditější než příslušné oxidy a hydroxidy v oxidačním stavu III.
	\end{itemize}
	\vfill
}

\frame{
	\frametitle{}
	\vfill
	\begin{itemize}
		\item Thallium i jeho sloučeniny jsou silně toxické.\footnote{\href{https://doi.org/10.1016/0020-711X(90)90254-Z}{Thallium in biochemistry}}
		\item Důvodem je zejména podobnost iontového poloměru \ce{Tl+} s \ce{K+}.
		\item Na rozdíl od iontů alkalických kovů má thallium odlišnou afinitu k~sloučeninám síry, takže působí jako blokátory redukčních systémů.
		\item Thallné soli se dobře vstřebávají i kůží.
		\item Při otravách se podává rozpustná forma berlínské modři, \ce{KFe[Fe(CN)6]}.\footnote{\href{https://doi.org/10.2165/00139709-200322010-00004}{Thallium Toxicity and the Role of Prussian Blue in Therapy}} Díky vyšší afinitě k iontům \ce{Tl+} funguje berlínská modř jako iontoměnič, ionty \ce{K+} uvolňuje a absorbuje ionty \ce{Tl+}.
		\item \ce{Tl+ + KFe[Fe(CN)6] -> K+ + TlFe[Fe(CN)6]}
	\end{itemize}
	\vfill
}

\section{Výskyt a získávání prvků}
\subsection{Gallium}
\frame{
	\frametitle{}
	\vfill
	\begin{itemize}
		\item V přírodě se nevyskytuje v čistém stavu.
		\item Zastoupení v zemské kůře je asi 16,9~ppm, to přibližně odpovídá dusíku, niobu, lithiu a olovu.
		\item Minerál \textit{gallit} (\ce{CuGaS2}) obsahuje 35~\% gallia, ale je příliš vzácný, takže se jako zdroj gallia nevyužívá.\footnote{\href{https://www.mindat.org/element/Gallium}{The mineralogy of Gallium}}
		\item Další minerály gallia jsou:
		\begin{itemize}
			\item \textit{Söhngeite} -- \ce{Ga(OH)3}
			\item \textit{Gallobeudantite} -- \ce{PbGa3(AsO4)(SO4)(OH)6}
			\item \textit{Tsumgallite} -- \ce{GaO(OH)}
			\item \textit{Galloplumbogummite} -- \ce{Pb(Ga,Al,Ge)3(PO4)2(OH)6}
		\end{itemize}
		\item V řádově nižších koncentracích se vyskytuje ve sfaleritu (ZnS), bauxitu (\ce{Al2O3.2 H2O}) a uhlí.
	\end{itemize}
	\vfill
}

\frame{
	\frametitle{}
	\vfill
	\begin{itemize}
		\item Gallium se získává jako vedlejší produkt při zpracování rud jiných kovů, převážně bauxitu (obsahuje 0,003-0,008~\% gallia) při výrobě hliníku. Menší množství se získávají také ze zinkových rud.\footnote{\href{https://doi.org/10.1016/j.resourpol.2015.11.005}{On the current and future availability of gallium}}
		\item Při výrobě hliníku Bayerovým procesem se gallium hromadí v roztoku hydroxidu sodného, z kterého je možné jej izolovat pomocí iontoměničů nebo zeolitů.\footnote{\href{https://doi.org/10.1007/s40831-019-00226-w}{Recovery of Gallium from Bauxite Residue Using Combined Oxalic Acid Leaching with Adsorption onto Zeolite HY}}
		\item Pro polovodičový průmysl se gallium vyrábí čištěním kovového gallia \textit{zonální tavbou}.
	\end{itemize}
	\begin{figure}
		\adjincludegraphics[width=.8\textwidth]{img/Gallium_sealed_in_vacuum_ampoule.jpg}
		\caption*{Kovové gallium.\footnote[frame]{\href{https://commons.wikimedia.org/wiki/File:6N_Gallium_sealed_in_vacuum_ampoule.jpg}{Zdroj: Alshaer666/Commons}}}
	\end{figure}
	\vfill
}

\frame{
	\frametitle{}
	\vfill
	\begin{itemize}
		\item Počáteční poměr Ga/Al je přibližně 1/5000 se postupně změní až na 1/300.
		\item Dále se koncentrace gallia zvyšuje elektrolýzou extraktů s využitím rtuťové elektrody.
		\item Získáme roztok obsahující gallitan sodný je pak elektrolyzován za vzniku kovového gallia.
		\item Další čištění gallia pro polovodičový průmysl se provádí reakcí s kyselinou a kyslíkem, následnou krystalizací a \textit{zonálním čištěním} (viz slide~\ref{zonalniTavba}).
	\end{itemize}
	\begin{figure}
		\adjincludegraphics[height=.3\textheight]{img/Hall-heroult.png}
		\caption*{Řez elektrolyzérem.\footnote[frame]{Zdroj: \href{https://commons.wikimedia.org/wiki/File:Hall-heroult-kk-2008-12-31.png}{Kashkan/Commons}}}
	\end{figure}
	\vfill
}

\frame{
	\frametitle{}
	\label{zonalniTavba}
	\vfill
	\begin{itemize}
		\item \textbf{Zonální tavení} se využívá na čištění krystalických látek.\footnote{\href{https://csacg.fzu.cz/func/viewpdf.php?file=2000_34Drapala.pdf}{Zonální tavba jako krystalizační a rafinační metoda}}
		\item Je založeno na opakovaném roztavení části krystalického materiálu.
		\item Po roztavení dochází k pohybu příměsí směrem dolů.
		\item Ochlazením vzniká z taveniny polykrystalického materiálu monokrystal.
		\item Opakováním tohoto procesu zvyšujeme čistotu materiálu.
	\end{itemize}
	\begin{figure}
		\adjincludegraphics[width=.6\textwidth]{img/Zone-refining.jpg}
		\caption*{Zonální tavba.\footnote[frame]{Zdroj: \href{https://commons.wikimedia.org/wiki/File:Zone-refining.jpg}{Dbuckingham42/Commons}}}
	\end{figure}
	\vfill
}

\frame{
	\frametitle{}
	\vfill
	\begin{columns}
		\begin{column}{.5\textwidth}
			\begin{figure}
				\adjincludegraphics[height=.55\textheight]{img/Si-crystal_floatingzone.jpg}
				\caption*{Výroba monokrystalu Si (100) zonální tavbou.\footnote[frame]{Zdroj: \href{https://commons.wikimedia.org/wiki/File:Si-crystal_floatingzone.jpg}{Matthias Renner/Commons}}}
			\end{figure}
		\end{column}
		\begin{column}{.5\textwidth}
			\begin{figure}
				\adjincludegraphics[height=.55\textheight]{img/Si-crystal_floatingzone_growing.jpg}
				\caption*{Výroba monokrystalu Si (100) zonální tavbou.\footnote[frame]{Zdroj: \href{https://commons.wikimedia.org/wiki/File:Si-crystal_floatingzone_growing.jpg}{Matthias Renner/Commons}}}
			\end{figure}

		\end{column}
	\end{columns}
	\vfill
}

\subsection{Indium}
\frame{
	\frametitle{}
	\vfill
	\begin{columns}
		\begin{column}{.7\textwidth}
			\begin{itemize}
				\item Jde o velmi vzácný prvek, jeho nedostatek může být kritický pro elektronický průmysl.
				\item Vytváří jen několik vzácných minerálů, ale žádný se nevyskytuje tak běžně, aby se jej vyplatilo komerčně těžit.
				\item Získává se elektrolyticky z popílků vznikajících při pražení sulfidických rud zinku a olova.\footnote[frame]{\href{https://doi.org/10.1016/S0892-6875(03)00168-7}{Processing of indium: a review}}
				\item Před rokem 1925 bylo celosvětově vyrobeno kolem 1 g kovového india.\footnote[frame]{\href{https://doi.org/10.1016/B978-0-7506-3365-9.50013-4}{Aluminium, Gallium, Indium and Thallium}}
				\item Mezi minerály india patří \textit{roquesite} (\ce{CuInS2}) a \textit{dzhalindite} (\ce{In(OH)3}).\footnote[frame]{\href{https://www.mindat.org/element/Indium}{The mineralogy of Indium}}
			\end{itemize}
		\end{column}

		\begin{column}{.4\textwidth}
			\begin{figure}
				\adjincludegraphics[width=\textwidth]{img/Indium_world_production.png}
				\caption*{Světová produkce india.\footnote[frame]{Zdroj: \href{https://commons.wikimedia.org/wiki/File:Indium_world_production.svg}{Con-struct/Commons}}}
			\end{figure}
		\end{column}
	\end{columns}
	\vfill
}

\subsection{Thallium}
\frame{
	\frametitle{}
	\vfill
	\begin{itemize}
		\item Jeho koncentrace v zemské kůře je asi 0,7 mg.kg$^{-1}$.
		\item Jako komerční zdroj thallia slouží rudy mědi, olova a zinku, ve kterých jsou stopová množství thallia.
		\item Známe několik minerálů obsahujících thallium:\footnote{\href{https://www.mindat.org/element/Thallium}{The mineralogy of Thallium}}
		\begin{itemize}
			\item Hutchinsonit -- \ce{(Tl,Pb)As5S9}
			\item Lorándite -- \ce{TlAsS2}
			\item Twinnite -- \ce{Pb0.8Tl0.1Sb1.3As0.8S4}
			\item Thalcusit -- \ce{Tl2Cu3FeS4}
			\item Crookesite -- \ce{Cu7(Tl,Ag)Se4}
		\end{itemize}
	\end{itemize}

	\begin{columns}
		\begin{column}{.3\textwidth}
			\begin{figure}
				\adjincludegraphics[height=.23\textheight]{img/Hutchinsonite-131710.jpg}
				\caption*{Hutchinsonit.\footnote[frame]{Zdroj: \href{https://commons.wikimedia.org/wiki/File:Hutchinsonite-131710.jpg}{Mindat/Commons}}}
			\end{figure}
		\end{column}
		\begin{column}{.3\textwidth}
			\begin{figure}
				\adjincludegraphics[height=.23\textheight]{img/Lorandite-Realgar-468096.jpg}
				\caption*{Lorandit s realgarem.\footnote[frame]{Zdroj: \href{https://commons.wikimedia.org/wiki/File:Lorandite,_Realgar-468096.jpg}{Mindat/Commons}}}
			\end{figure}
		\end{column}
	\end{columns}
	\vfill
}

\frame{
	\frametitle{}
	\vfill
	\begin{itemize}
		\item Výroba kovového thallia je poměrně obtížná, získává se z odpadů při zpracování rud jiných kovů, nejčastěji zinku a olova.\footnote[frame]{\href{https://pubs.usgs.gov/periodicals/mcs2022/mcs2022-thallium.pdf}{USGS -- Thallium}}
		\item Surovina se rozpustí v teplé zředěné kyselině sírové, tím se oddělí nerozpustný síran olovnatý, \ce{PbSO4}.
		\item Přídavkem HCl se vysráží TlCl, který se několikrát přečišťuje.
		\item Thallium se získá elektrolýzou roztoku \ce{Tl2SO4} v kyselině sírové pomocí elektrod z Pt nebo nerezové oceli.
		\item \ce{Tl+ + e- -> Tl}
		\item Thallium se taví v atmosféře vodíku při teplotě 350--400~$^\circ$C.\footnote[frame]{GREENWOOD, N. N. a Alan EARNSHAW. \textit{Chemie prvků.} Praha: Informatorium, 1993. ISBN 80-854-2738-9. s. 269}
	\end{itemize}
	\begin{figure}
		\adjincludegraphics[height=.25\textheight]{img/Thallium.jpg}
	\end{figure}
	\vfill
}

\section{Využití prvků}
\subsection{Gallium}
\frame{
	\frametitle{}
	\begin{columns}
	\begin{column}{.7\textwidth}
	\vfill
	\begin{itemize}
	\item Oproti hliníku je produkce a využití Ga, In a Tl řádově menší.
	\item Hlavní využití gallia je v polovodičovém průmyslu, kde se využívá kov s vysokou čistotu ($>$99,9999~\%).
	\item Slitiny gallia mají nízkou teplotu tání, eutektická slitina gallia, india a cínu, označovaná jako \textit{galinstan}, se využívá v lékařských teploměrech. Má teplotu tání $-$19~$^\circ$C.
	\item Gallium lze použít jako náhradu rtuti při konstrukci kapalinových zrcadel, např. v teleskopech.\footnote[frame]{\href{https://iopscience.iop.org/article/10.1086/133893}{Gallium Liquid Mirrors}}
	\item Soli izotopu \ce{^6^7Ga} se využívají jako radiofarmaka.\footnote[frame]{\href{https://www.lf3.cuni.cz/3LF-839-version1-klinicke_pouziti_radiofarmak_v_nuklearni_medicine.pdf}{Klinické použití radiofarmak v nukleární medicíně}}
	\end{itemize}
	\vfill
	\end{column}
	\begin{column}{.4\textwidth}
		\begin{figure}
			\adjincludegraphics[width=\textwidth]{img/Liquid_Mirror_Telescope.jpg}
			\caption*{Kapalinové zrcadlo.\footnote[frame]{Zdroj: \href{https://commons.wikimedia.org/wiki/File:Liquid_Mirror_Telescope.jpg}{NASA/Commons}}}
		\end{figure}
	\end{column}
	\end{columns}
}

\subsection{Gallium -- polovodičový průmysl}
\frame{
	\frametitle{}
	\vfill
	\begin{itemize}
		\item Nejdůležitější sloučeniny gallia v polovodičovém průmyslu jsou nitrid gallitý (GaN) a arsenid gallitý (GaAs).
		\item GaN je polovodič III/V používaný pro konstrukci PN přechodů pro LED, displeje a MESFET tranzistory (metal–semiconductor field-effect transistor).\footnote[frame]{\href{https://doi.org/10.1016/S0026-2692(01)00062-3}{A complete analytical model of GaN MESFET for microwave frequency applications}}
		\item Má strukturu wurtzitu, atomy gallia i dusíku mají tetraedrickou koordinaci.
		\item Krystaly se připravují pomocí epitaxe z molekulárních svazků, např. pomocí směsi trimethylgallanu a amoniaku na křemíkový substrát.
		\item \ce{(CH3)3Ga + NH3 -> GaN + CH4}
		\item Práškový se vyrábí reakcí gallia nebo oxidu gallitého s amoniakem.
		\item \ce{2 Ga + 2 NH3 -> 2 GaN + 3 H2}
		\item \ce{Ga2O3 + 2 NH3 -> 2 GaN + 3 H20}
	\end{itemize}
	\vfill
}

\frame{
	\frametitle{}
	\vfill
	\begin{columns}
		\begin{column}{.5\textwidth}
			\begin{figure}
				\adjincludegraphics[width=\textwidth]{img/Blue_LED_and_Reflection.jpg}
				\caption*{LED diody.\footnote[frame]{Zdroj: \href{https://commons.wikimedia.org/wiki/File:Blue_LED_and_Reflection.jpg}{Alexofdodd/Commons}}}
			\end{figure}
		\end{column}
		\begin{column}{.5\textwidth}
			\begin{figure}
				\adjincludegraphics[width=\textwidth]{img/GaN_Wurtzite_polyhedra.png}
				\caption*{Struktura wurtzitu, ZnS.\footnote[frame]{Zdroj: \href{https://commons.wikimedia.org/wiki/File:GaN_Wurtzite_polyhedra.png}{Solid State/Commons}}}
			\end{figure}
		\end{column}
	\end{columns}
	\vfill
}

\frame{
	\frametitle{}
	\vfill
	\begin{itemize}
		\item GaAs je polovodič III/V používaný pro konstrukci PN přechodů v různých typech tranzistorů, díky vlastnostem GaAs mohou tyto tranzistory pracovat až do frekvence 250~GHz.\footnote[frame]{\href{https://www.jstor.org/stable/24979446}{Gallium Arsenide Transistors}}
		\item Využívá se také při konstrukci fotovoltaických článků s vysokou účinností, např. pro vesmírné sondy.
		\item Má strukturu sfaleritu.
		\item Vyrábí se několika metodami:
		\begin{itemize}
			\item VGF (Vertical Gradient Freeze) procesem -- tavenina je ve válcovém kelímku postupně ochlazována.\footnote[frame]{\href{https://arxiv.org/pdf/2002.11447.pdf}{Control of the Vertical Gradient Freeze crystal growth process via backstepping}}
			\item Pomocí Czochralskiho metody.
			\item MOCVD:
			\item \ce{Ga(CH3)3 + AsH3 -> GaAs + 3 CH4}
		\end{itemize}
	\end{itemize}
	\vfill
}

\frame{
	\frametitle{}
	\vfill
	\begin{columns}
		\begin{column}{.5\textwidth}
			\begin{figure}
				\adjincludegraphics[width=\textwidth]{img/Gallium-arsenide-unit-cell-3D-balls.png}
				\caption*{Struktura GaAs.\footnote[frame]{Zdroj: \href{https://commons.wikimedia.org/wiki/File:Gallium-arsenide-unit-cell-3D-balls.png}{Benjah-bmm27/Commons}}}
			\end{figure}
		\end{column}
		\begin{column}{.5\textwidth}
			\begin{center}
				\begin{figure}
					\adjincludegraphics[height=.9\textwidth,angle=90]{img/MidSTAR-1.jpg}
					\caption*{Satelit MidSTAR-1 se solárními panely z GaAs.\footnote[frame]{Zdroj: \href{https://commons.wikimedia.org/wiki/File:MidSTAR-1.jpg}{United States Naval Academy/Commons}}}
				\end{figure}
			\end{center}
		\end{column}
	\end{columns}
	\vfill
}

\subsection{Indium}
\frame{
	\frametitle{}
	\vfill
	\begin{columns}
		\begin{column}{.75\textwidth}
			\begin{itemize}
				\item Dříve se využívalo jako ochrana ložisek proti mechanickému opotřebování a proti korozi.
				\item Slitiny s nízkou teplotou tání.
				\item Ve vakuové technice se využívají pájky s obsahem india ke spojování kovových a nekovových součástí aparatur.\footnote[frame]{\href{https://www.youtube.com/watch?v=h8DkUlI-Ngs}{How to make Indium seals}}
				\item Je to důležitý prvek v polovodičové technice:
				\begin{itemize}
					\item ITO - Indium Tin Oxide.
					\item Pájení polovodičových součástek za nízkých teplot.
					\item Výroba spojů v P-N-P přechodech tranzistorů.
				\end{itemize}
				\item Je součástí moderátorových tyčí v některých jaderných reaktorech.\footnote[frame]{\href{https://www.indium.com/blog/indium-alloy-for-use-in-control-rods-for-nuclear-reactors.php}{Indium Alloy for use in Control Rods for Nuclear Reactors}}
			\end{itemize}
		\end{column}
		\begin{column}{.3\textwidth}
			\begin{figure}
				\adjincludegraphics[width=\textwidth]{img/Indium_wire.jpg}
				\caption*{Indiový drát pro pájení.\footnote[frame]{Zdroj: \href{https://commons.wikimedia.org/wiki/File:Indium_wire.jpg}{Dschwen/Commons}}}
			\end{figure}
		\end{column}
	\end{columns}
	\vfill
}

\subsubsection{Nízkotající slitiny}
\frame{
	\frametitle{}
	\textbf{Nízkotající slitiny}
	\begin{tabular}{|l|l|l|}
		\hline
		\textbf{Slitina} & \textbf{Obchodní název} & \textbf{Teplota tání [$^\circ$C]} \\
		\hline
		\textbf{Bi45Pb23Sn8In19Cd5} & Slitina 47  &  47 \\
		\hline
		\textbf{Bi49Pb18Sn12In21} & Slitina 58  & 58 \\
		\hline
		Bi50Pb27Sn13Cd & Woodův kov 1 & 68-72 \\
		\hline
		Bi50Pb25Sn12Cd & Woodův kov 2 & 60-64 \\
		\hline
		Bi50Sn25Pb & Roseův kov & 92-96 \\
		\hline
		Bi55Pb32Sn & Molotův kov & 96-98 \\
		\hline
		Bi74Pb7Sn & Biola 1 & 104-463 \\
		\hline
		Bi50Pb43Cd & Biola 3 & 80-84 \\
		\hline
		Bi8Sn57Pb & Stabia 1 & 139-178 \\
		\hline
		Pb45Bi10Sn & Stabia 4 & 97-169 \\
		\hline
		Pb25Bi25Sn & Stabia 6 & 97-161 \\
		\hline
		Pb73Bi23Sn3Zn & Plumbia 3 & 183-224 \\
		\hline
		Pb57Bi8Sn & Plumbia 5 & 174-214 \\
		\hline
	\end{tabular}
}

\subsection{Indium/gallium -- kapalný kov}
\frame{
	\frametitle{}
	\begin{itemize}
		\item Nízkotající slitiny india s galliem lze využít jako vodivé spoje, schopné samostatně opravovat poškození vodivé cesty.\footnote[frame]{\href{https://dx.doi.org/10.1126/sciadv.abd0202}{Heterogeneous integration of rigid, soft, $\ldots$}}
		\item Pokud dojde vlivem deformace k porušení celistvosti vodivého spoje, spoj se obnoví, jakmile přestane síla působit.
		\item Výhodným materiálem je eutektická slitina india a gallia. Problémem je cena a toxicita.\footnote[frame]{\href{https://www.sigmaaldrich.com/catalog/product/aldrich/495425}{Gallium–Indium eutectic}}
	\end{itemize}
	\begin{center}
		\adjincludegraphics[height=45mm]{img/GaIn-alloy.png}
	\end{center}
	\vfill
}

\subsection{Indium -- ITO}
\frame{
	\frametitle{}
	\begin{itemize}
		\item Směs oxidů inditého a cíničitého, přibližný vzorec je \ce{(In2O3)0_{.9} . (SnO2)_{0.1}}
		\item Je to nejrozšířenější transparentní a vodivý oxid.
		\item Lze z něj snadno připravit tenký, vodivý film.\footnote[frame]{\href{https://www.researchgate.net/publication/321172668_Structural_Optical_and_Electrical_Properties_of_ITO_Thin_Films}{Structural, Optical and Electrical Properties of ITO Thin Films}}
		\item Poměr transparentnosti a vodivosti filmu lze řídit tloušťkou filmu.
		\begin{itemize}
			\item Tenký film je vysoce průhledný, ale má vyšší odpor.
			\item Silnější vrstvy jsou méně průhledné, ale mají nižší elektrický odpor.
		\end{itemize}
		\item Lze vyrobit vysoce průhledný film s dostatečným elektrickým odporem, např. pro odledování oken letadel.\footnote[frame]{\href{https://phys.org/news/2019-08-defrosting-surfaces-seconds.html}{Defrosting surfaces in seconds}}
	\end{itemize}
	\vfill
}

\frame{
	\frametitle{}
	\vfill
	\begin{columns}
		\begin{column}{.5\textwidth}
			\begin{figure}
				\adjincludegraphics[height=.55\textheight]{img/Kristallstruktur_Lanthanoid-C-Typ.png}
				\caption*{Krystalová struktura ITO.\footnote[frame]{Zdroj: \href{https://commons.wikimedia.org/wiki/File:Kristallstruktur_Lanthanoid-C-Typ.png}{Orci/Commons}}}
			\end{figure}
		\end{column}
		\begin{column}{.5\textwidth}
			\begin{figure}
				\adjincludegraphics[height=.55\textheight]{img/lossy-page1-834px-ITO_grains_on_glass.jpg}
				\caption*{Snímek filmu z ITO.\footnote[frame]{Zdroj: \href{https://commons.wikimedia.org/wiki/File:ITO_grains_on_glass.tif}{Topliuchao/Commons}}}
			\end{figure}
		\end{column}
	\end{columns}
	\vfill
}

\subsection{Thalium}
\frame{
	\frametitle{}
		\vfill
		\begin{itemize}
			\item Dříve se používal síran thallný jako jed na krysy a mravence.\footnote[frame]{\href{https://www.ebi.ac.uk/chebi/searchId.do?chebiId=CHEBI:81836}{CHEBI:81836 - thallium sulfate}}
			\item Směsné halogenidy thallné se používají jako optická skla pro infračervenou spektroskopii (KRS-5, KRS-6), jsou dobře propustná pro IR záření a nerozpustná ve vodě.
		\end{itemize}
	\vfill
	\begin{figure}
		\adjincludegraphics[height=.4\textheight]{img/KRS5_Tallium_Bromide_Iodide_ingots_Crystaltechno.jpg}
		\caption*{Ingoty z KRS-5.\footnote[frame]{Zdroj: \href{https://commons.wikimedia.org/wiki/File:6_KRS5_Tallium_Bromide_Iodide_ingots_Crystaltechno.jpg}{Crystaltechno/Commons}}}
	\end{figure}
}

\frame{
	\frametitle{}
	\begin{columns}
		\begin{column}{.7\textwidth}
			\vfill
			\begin{itemize}
				\item Slitina thallia se rtutí vytváří eutektikum (8,5~\% Tl) s teplotou tání $-60 ^\circ$C, tj. dvacet stupňů pod bodem tání rtuti.
				\item Sulfid a selenid thallný (\ce{Tl2S}, \ce{Tl2Se}) se využívá při konstrukci \textit{bolometrů} pro detekci IR záření.\footnote[frame]{\href{https://doi.org/10.1364/AO.10.001003}{Cashman Thallous Sulfide Cell}}, \footnote[frame]{\href{https://doi.org/10.1364/AO.16.002942}{Thallium selenide infrared detector}}
				\item Sulfid thallný slouží ke konstrukci fotorezistorů pro infračervenou oblast.
				\item Je součástí vysokoteplotních supravodičů -- \ce{Tl2Ba2CuO6}, \ce{TlBaCaCuO}.\footnote[frame]{\href{https://doi.org/10.1038/332138a0}{Bulk superconductivity at 120 K in the Tl–Ca/Ba–Cu–O system}}
			\end{itemize}
			\vfill
		\end{column}
		\begin{column}{.35\textwidth}
			\begin{figure}
				\adjincludegraphics[width=\textwidth]{img/TBCCO-2223_structure_schema_en.png}
				\caption*{Struktura \ce{TlBaCaCuO}.\footnote[frame]{Zdroj: \href{https://commons.wikimedia.org/wiki/File:TBCCO-2223_structure_schema_en.svg}{Winiar/Commons}}}
			\end{figure}
		\end{column}
	\end{columns}
}

\section{Sloučeniny}
\subsection{Hydridy}
\frame{
	\frametitle{}
	\vfill
	\begin{itemize}
		\item Gallan neboli hydrid gallitý (\textbf{\ce{GaH3}}) existuje ve formě dimeru (digallanu) \textbf{\ce{Ga2H6}}.
		\item Monomer lze detekovat pouze za nízkých teplot (3,5~K) nebo ve formě aduktů, např.:
		\item \ce{Ga2H6 + 4 NMe3 ->[-95 $^\circ$C] 2 (NMe3)2*GaH3}
		\item Digallan lze připravit reakcí chloridu gallitého s trimethylsilanem při teplotě $-23~^\circ$C a následnou redukcí produktu:
		\item \ce{Ga2Cl6 + 4 Me3SiH -> (H2GaCl)2 + 4 Me3SiCl}
		\item \ce{(H2GaCl)2 + 2 LiGaH4 -> Ga2H6 + LiCl}
		\item Komplexní tetrahydridogallitany (analoga \ce{[AlH4]-}) jsou stálé sloučeniny, \ce{LiGaH4} má poměrně silné redukční účinky.
	\end{itemize}
	\vfill
	\begin{figure}
		\adjincludegraphics[height=0.15\textheight]{img/Ga2H6.png}
	\end{figure}
}

\frame{
	\frametitle{}
	\vfill
	\begin{itemize}
		\item Indigan neboli hydrid inditý (\textbf{\ce{InH3}}) vytváří kovalentní polymerní síť, je stabilní pouze za nízkých teplot (pod $-90\ ^\circ$C). Chová se jako Lewisova kyselina, vytváří adukty s ligandy (1:1 nebo 1:2).\footnote[frame]{\href{https://doi.org/10.1039/B107285B}{The stabilisation and reactivity of indium trihydride complexes}}
		\item \ce{InH3 + L -> InH3L}
		\item Neplést s uhlovodíkem \textit{indanem}.\footnote[frame]{\href{https://pubchem.ncbi.nlm.nih.gov/compound/indane}{Indane}}
	\end{itemize}
	\vfill
	\begin{figure}
		\adjincludegraphics[height=0.2\textheight]{img/Indane.png}
		\adjincludegraphics[height=0.4\textheight]{img/Cy3PInH3.png}
	\end{figure}
}

\frame{
	\frametitle{}
	\vfill
	\begin{itemize}
		\item Thallan neboli hydrid thallitý (\textbf{\ce{TlH3}}) byl izolován pouze za nízkých teplot v matrici z inertního plynu reakcí thallia s vodíkem.\footnote[frame]{\href{https://doi.org/10.1021/ic950411u}{Are the Compounds \ce{InH3} and \ce{TlH3} Stable Gas Phase or Solid State Species?}}
		\item Navržený mechanismus vzniku:\footnote[frame]{\href{https://doi.org/10.1021/jp0498973}{Infrared Spectra of Thallium Hydrides in Solid Neon, Hydrogen, and Argon}}
	\end{itemize}
	\begin{columns}
		\begin{column}{.5\textwidth}
			\begin{align*}
				\ce{Tl + H2 &-> TlH + H \\
					Tl + H2 &-> TlH2 \\
					TlH + H2 &-> TlH3 \\
					TlH2 + H &-> TlH3}
			\end{align*}
		\end{column}

		\begin{column}{.5\textwidth}
			\begin{figure}
				\adjincludegraphics[height=0.3\textheight]{img/Thallane-3D-vdW.png}
				\caption*{Model molekuly \ce{TlH3}.\footnote[frame]{Zdroj: \href{https://commons.wikimedia.org/wiki/File:Thallane-3D-vdW.png}{Hoa112008/Commons}}}
			\end{figure}
		\end{column}
	\end{columns}

	\vfill
}


\subsection{Oxidy a hydroxidy}
\frame{
	\frametitle{}
	\vfill
	\begin{itemize}
		\item Ve skupině roste zásaditý charakter směrem dolů.
		\item Oxid boritý je kyselý.
		\item \ce{B2O3 + 3 H2O -> 2 H3BO3}
		\item Oxidy hliníku a gallia jsou amfoterní.
		\item \ce{Ga2O3 + 6 HCl -> 2 GaCl3 + 3 H2O}
		\item \ce{Ga2O3 + 2 NaOH -> 2 NaGaO2 + H2O}
		\item Oxidy india a thallia jsou zásadité.
		\item Oxid thallný je rozpustný ve vodě, vzniká hydroxid, který je srovnatelně silnou zásadou jako KOH.
		\item \ce{Tl2O + H2O -> 2 TlOH}
	\end{itemize}
	\vfill
}

\frame{
	\frametitle{}
	\vfill
	\begin{itemize}
		\item Oxid gallitý vytváří pět krystalických modifikací ($\alpha-\epsilon$).
	\end{itemize}
	\begin{figure}
		\adjincludegraphics[width=0.8\textwidth]{img/Ga2O3.png}
	\end{figure}
	\vfill
}

\frame{
	\frametitle{}
	\vfill
	\begin{itemize}
		\item Nejstabilnější modifikací je \ce{$\beta$-Ga2O3}. Ionty kyslíku zaujímají pozice odpovídající deformovanému nejtěsnějšímu kubickému uspořádání. Gallité ionty zaujímají dvě rozdílné pozice.
		\begin{itemize}
			\item Ga(I) -- deformovaný tetraedr \ce{GaO4}
			\item Ga(II) -- deformovaný oktaedr \ce{GaO6}
		\end{itemize}
		\item Polovina gallitých iontů má nižší koordinační číslo, z toho důvodu je hustota nižší o 10~\% oproti modifikaci $\alpha$.
	\end{itemize}
	\begin{figure}
		\adjincludegraphics[width=0.7\textwidth]{img/Kristallstruktur_Galliumoxid.png}
		\caption*{Krystalová struktura \ce{$\beta$-Ga2O3}.\footnote[frame]{Zdroj: \href{https://commons.wikimedia.org/wiki/File:Kristallstruktur_Galliumoxid.png}{Orci/Commons}}}
	\end{figure}
	\vfill
}

\frame{
	\frametitle{}
	\vfill
	\begin{itemize}
		\item \ce{$\alpha$-Ga2O3} vzniká kalcinací \ce{$\beta$-Ga2O3} při teplotě 1000~$^\circ$C za vysokého tlaku.
		\item \ce{$\beta$-Ga2O3} lze připravit termickým rozkladem solí (dusičnanů, octanů, $\ldots$) při teplotách nad 1000~$^\circ$C.
		\item \ce{$\gamma$-Ga2O3} vzniká prudkým ohřevem hydroxidu na teplotu 400--500~$^\circ$C.
		\item \ce{$\delta$-Ga2O3} připravíme rozkladem dusičnanu při teplotě 250~$^\circ$C.
		\item \ce{$\epsilon$-Ga2O3} získáme zahříváním \ce{$\delta$-Ga2O3} na teplotu 550~$^\circ$C.
	\end{itemize}
	\vfill
}

\frame{
	\frametitle{}
	\vfill
	\begin{itemize}
		\item \textit{Oxid gallitý je amfoterní}, za vysoké teploty reaguje s alkalickými kovy za vzniku \ce{NaGaO2}, s oxidy Mg, Zn, Ni, Co a Cu poskytuje spinely \ce{MGa2O4}.
		\item Rozpouštěním v kyselině chlorovodíkové vzniká chlorid gallitý:
		\item \ce{Ga2O3 + 6 HCl -> 2 GaCl3 + 3 H2O}
		\item Redukcí vodíkem nebo kovovým galliem vzniká oxid gallný.\footnote[frame]{\href{https://pubchem.ncbi.nlm.nih.gov/compound/16702109}{Gallium(I) oxide}}
		\item \ce{Ga2O3 + 2 H2 -> Ga2O + 2 H2O}
		\item \ce{Ga2O3 + 4 Ga -> 3 Ga2O}
		\item \textit{Oxid gallný} lze dále připravit reakcí kovového gallia s oxidem uhličitým:
		\item \ce{2 Ga + CO2 ->[vakuum, 850 $^\circ$C] Ga2O + CO}
		\item Je to hnědočerná diamagnetická sloučenina, na suchém vzduchu je stabilní.
	\end{itemize}
	\vfill
}

\frame{
	\frametitle{}
	\vfill
	\begin{itemize}
		\item \textit{Spinely}
		\item Velká skupina sloučenin s krystalovou strukturou příbuznou minerálu spinelu \ce{MgAl2O4}.
		\item Mají obecný vzorec \ce{AB2X4}.
		\item Jejich elementární buňka obsahuje 32 atomů kyslíku v nejtěsnějším krychlovém uspořádání, \ce{A8B16X32}.
		\item Atomy A obsazují vrcholy tetraedru a atomy B vrcholy oktaedru.
		\item Aniontem je zpravidla O, S, Se nebo Te.
		\item Nejběžnější stechiometrie spinelu jsou:
		\item $\mbox{A}^{\mbox{II}}\mbox{B}^{\mbox{III}}_2\mbox{O}_4$, $\mbox{A}^{\mbox{IV}}\mbox{B}^{\mbox{II}}_2\mbox{O}_4$ a $\mbox{A}^{\mbox{VI}}\mbox{B}^{\mbox{I}}_2\mbox{O}_4$
		\item Můžeme se ale potkat i s exotičtějšími: \ce{NiLi2F4}, \ce{ZnK2(CN)4}.
	\end{itemize}
	\vfill
}

\frame{
	\frametitle{}
	\vfill
	\begin{figure}
		\adjincludegraphics[width=.9\textwidth]{img/AB2O4_spinel.jpg}
		\caption*{Krystalová struktura spinelu \ce{AB2O4}.\footnote[frame]{Zdroj: \href{https://commons.wikimedia.org/wiki/File:AB2O4_spinel.jpg}{Tem5psu/Commons}}}
	\end{figure}
	\vfill
}

\frame{
	\frametitle{}
	\vfill
	\begin{itemize}
		\item \textit{Oxid inditý} (\ce{In2O3}) je v amorfním stavu nerozpustný ve vodě, ale rozpouští se v kyselinách. Krystalický je rozpustný i ve vodě. Lze jej připravit zahřívání vhodné indité soli -- hydroxidu, uhličitanu, dusičnanu nebo síranu.
		\item \textit{Hydroxid inditý} (\ce{In(OH)3}) je amfoterní hydroxid, lépe se rozpouští v kyselinách než v zásadách. Lze jej připravit srážením vodného roztoku \ce{InCl3} za teploty varu amoniakem a následným stárnutím sraženiny (gelu).\footnote[frame]{\href{https://doi.org/10.1007/s10973-005-0963-4}{Preparation and thermal decomposition of indium hydroxide}}
		\begin{itemize}
			\item Je hlavním prekurzorem pro oxid inditý.
			\item Má kubickou strukturu.
			\item Je nerozpustný ve vodě.
			\item Za vyšší teploty a tlaku přechází na InO(OH).
			\item V přírodě se vyskytuje jako vzácný minerál \textit{dzhalindite}.\footnote[frame]{\href{http://webmineral.com/data/Dzhalindite.shtml}{Dzhalindite Mineral Data}}
		\end{itemize}
	\end{itemize}
	\vfill
}

\frame{
	\frametitle{}
	\vfill
	\begin{figure}
		\adjincludegraphics[height=.7\textheight]{img/Indium(III)-oxide.jpg}
		\caption*{Oxid inditý}
	\end{figure}
	\vfill
}

\frame{
	\frametitle{}
	\begin{columns}
		\begin{column}{0.65\textwidth}
	\vfill
	\label{ITO}
	\begin{itemize}
		\item \textit{ITO -- oxid indito-cíničitý}
		\item Jeho přibližný vzorec je \ce{(In2O3)_{0.9} . (SnO2)_{0.1}}.
		\item Hmotnostní složení: 74 \% In, 18 \% \ce{O2}, and 8 \% Sn.
		\item Je to jeden z nejpoužívanějších průhledných vodivých filmů.
		\item Tloušťkou filmu lze nastavit jak jeho průhlednost, tak vodivost.
		\item Využívá se pro konstrukci LCD, OLED, solárních článků, atd.
		\item Průhlednosti se využívá při potahování oken, např. u letadel slouží k odstraňování námrazy (po připojení elektrického napětí dochází ke generování tepla).
	\end{itemize}
	\vfill
		\end{column}
		\begin{column}{0.4\textwidth}
			\begin{figure}
				\adjincludegraphics[width=\textwidth]{img/LHcockpitWindow.jpg}
			\caption*{Airbus A319, skla jsou opatřena tenkým filmem z ITO.\footnote[frame]{Zdroj: \href{https://commons.wikimedia.org/wiki/File:LHcockpitWindow.jpg}{Etan J. Tal/Commons}}}
			\end{figure}
		\end{column}
	\end{columns}
}

\frame{
	\frametitle{}
	\vfill
	\begin{itemize}
		\item \textit{Oxid thallitý} (\ce{Tl2O3}) má hnědou až černou barvu.\footnote[frame]{\href{https://doi.org/10.1002/14356007.a26_607}{Thallium and Thallium Compounds}}
		\begin{itemize}
			\item Vzniká zahříváním oxidu thallného na teploty nad 100~$^\circ$C nebo reakcí solí thallitých s amoniakem nebo KOH.
			\item Další možností přípravy je oxidace chloridu thalného pomocí chlornanu.
			\item \ce{2 TlCl + 2 NaClO + 2 NaOH -> Tl2O3 + 4 NaCl + H2O}
			\item Při teplotách nad 500~$^\circ$C sublimuje a částečně se rozkládá.
			\item \ce{Tl2O3 <=> Tl2O + O2}
			\item Ochotně reaguje (i za laboratorní teploty) se sírou nebo sulfanem za vzniku sulfidu thallného.
			\item \ce{2 Tl2O3 + 5 S -> 2 Tl2S + 3 SO2}
			\item \ce{3 Tl2O3 + 5 H2S -> 3 Tl2S + 5 H2O + 2 SO2}
		\end{itemize}
		\item \textit{Hydroxid thallitý} (\ce{Tl(OH)3}) se zatím nepodařilo připravit.
	\end{itemize}
\vfill
}

\frame{
	\frametitle{}
	\vfill
	\begin{itemize}
		\item \textit{Oxid thallný} (\ce{Tl2O}) tvoří černé orthorombické krystaly.\footnote[frame]{\href{https://doi.org/10.1002/zaac.19713810305}{Zur Darstellung und Kristallstruktur von Tl$_2$O}}
		\begin{itemize}
			\item Lze jej připravit kalcinací uhličitanu thallného (\ce{Tl2CO3}) v inertní atmosféře.
			\item Vzniká oxidací na suchém vzduchu při teplotách nad 100~$^\circ$C.
			\item Je hygroskopický, snadno se hydratuje za vzniku TlOH.
		\end{itemize}
		\item \textit{Hydroxid thallný} (TlOH) má podobné vlastnosti jako hydroxidy alkalických kovů. Je bílý, po expozici světlu šedne.
		\begin{itemize}
			\item Je velmi dobře rozpustný ve vodě a jde o silnou zásadu.
			\item Mimo hydratace oxidu thallného vzniká i reakcí kovového thallia se vzdušnou vlhkostí:
			\item \ce{2 Tl + H2O + O2 -> 2 TlOH}\footnote[frame]{\href{https://doi.org/10.1021/ja01335a001}{A New Method for the Preparation of Thallous Hydroxide}}
			\item Podobně jako hydroxidy alkalických kovů reaguje se vzduchem za vzniku uhličitanu.
			\item Roztokem alkalického sulfidu se sráží za vzniku sulfidu thallného.
			\item \ce{2 TlOH + Na2S -> Tl2S + 2 NaOH}
		\end{itemize}
	\end{itemize}
	\vfill
}

\subsection{Chalkogenidy}
\frame{
	\frametitle{}
	\vfill
	\begin{itemize}
		\item Chalkogenidů gallia, india a thallia známe poměrně mnoho.
		\item \textit{Sulfid gallitý} (\ce{Ga2S3}) je žlutá pevná látka s polovodivými vlastnostmi.
		\item Vytváří čtyři polymorfní formy:
		\begin{itemize}
			\item $\alpha$ -- hexagonální, žlutá forma.
			\item $\alpha'$ -- monoklinická.
			\item $\beta$ -- hexagonální.
			\item $\gamma$ -- kubická, defektní spinelová struktura.
		\end{itemize}
		\item Můžeme jej připravit přímou reakcí prvků nebo zahříváním gallia v proudu sulfanu.
		\item \ce{2 Ga + 3 S -> Ga2S3}
		\item \ce{4 Ga + 6 H2S ->[950 $^\circ$C] 2 Ga2S3 + 3 H2}
		\item S vodným roztokem \ce{K2S} poskytuje klastr \ce{K8Ga4S10}, který obsahuje anion \ce{[Ga4S10]^{8-}} se strukturou adamantanu.
	\end{itemize}
	\vfill
}

\frame{
	\frametitle{}
	\vfill
	\begin{figure}
			\adjincludegraphics[height=0.75\textheight]{img/Ga4S10.png}
			\caption*{Krystalová struktura aniontu \ce{[Ga4S10]^{8-}}}
	\end{figure}
	\vfill
}

\frame{
	\frametitle{}
	\vfill
	\begin{itemize}
		\item \textit{Sulfid gallnatý} (\ce{GaS}) je žlutá pevná látka s teplotou tání 970~$^\circ$C.
		\item Má hexagonální vrstevnatou strukturu, ke každému atomu gallia jsou koordinovány tři atomy síry a jeden gallia (obsahují jednotku \ce{Ga$_2^{4+}$}).
		\item Podobnou strukturu mají i GaSe, GaTe, InS a InSe.
	\end{itemize}
	\begin{figure}
		\begin{figure}
			\adjincludegraphics[width=0.6\textwidth]{img/GaSstructure.jpg}
			\caption*{Wurtzitová struktura GaS.\footnote[frame]{Zdroj: \href{https://commons.wikimedia.org/wiki/File:GaSstructure.jpg}{Materialscientist/Commons}}}
		\end{figure}
	\end{figure}
	\vfill
}

\frame{
	\frametitle{}
	\vfill
	\begin{itemize}
		\item \textit{Selenid gallnatý} (\ce{GaSe}) je hnědá pevná látka s teplotou tání 960~$^\circ$C.
		\item Nanočástice se připravují reakcí trimethylgallia s trioctylfosfinselanem.\footnote[frame]{\href{https://doi.org/10.1021/nl015641m}{Synthesis of Highly Luminescent GaSe Nanoparticles}}
		\item \textit{Selenid gallitý} (\ce{Ga2Se3}) je načervenalá pevná látka.
		\item Lze jej připravit syntézou z prvků, ve vodě pomalu hydrolyzuje.
		\item Má mírné oxidační účinky.
		\item \textit{Tellurid gallnatý} (\ce{GaTe}) je černá pevná látka, připravuje se pomocí MOCVD.\footnote[frame]{MOCVD -- Metal-Organic Chemical Vapor Deposition}
		\item \textit{Tellurid gallitý} (\ce{Ga2Te3}) tvoří černé krystaly, nerozpustné ve vodě.
		\item Má polovodivé vlastnosti, ve formě tenkého filmu jde o perspektivní materiál pro solární články.\footnote[frame]{\href{https://doi.org/10.1109/ICIPRM.2016.7528605}{Growth and solar cell applications of \ce{AgGaTe2} layers by closed space sublimation using the mixed source of \ce{Ag2Te} and \ce{Ga2Te3}}}
	\end{itemize}
	\vfill
}

\frame{
	\frametitle{}
	\vfill
	\begin{itemize}
		\item \textit{Sulfid inditý} (\ce{In2S3}) je pevná látka, zapáchající po zkažených vejcích.
		\item Byl připraven už roku 1863 jako první sloučenina india.\footnote[frame]{\href{https://doi.org/10.1002/prac.18630890156}{Vorläufige Notiz über ein neues Metal}}
		\item Známe tři polymorfní struktury:
		\begin{enumerate}
			\item $\alpha$-\ce{In2S3} -- žlutá, kubická modifikace.
			\item $\beta$-\ce{In2S3} -- červená modifikace s defektní spinelovou strukturou. Stabilní modifikace za laboratorní teploty
			\item $\gamma$-\ce{In2S3} -- má vrstevnatou strukturu, je nestabilní.
		\end{enumerate}
		\item Připravuje se přímou syntézou z prvků.
		\item \ce{^{113m}In2S3} je radioaktivní (T$_\frac{1}{2}$ = 1,66 h) a využívá se jako kontrastní látka v medicíně.\footnote[frame]{\href{https://doi.org/10.1148/92.7.1453}{Radioactive Albumin Microspheres for Studies of the Pulmonary Circulation}}
		\item Nanočástice \ce{In2S3} dopované europiem se využívají jako luminiscenční materiály. Při složení \ce{In_{1.8}Eu_{0.2}S3} emitují zelené světlo (510~nm), zatímco \ce{In_{1.6}Eu_{0.4}S3} emitují modré světlo (425~nm).\footnote[frame]{\href{https://doi.org/10.1021/jp048107m}{Full-Color Emission from \ce{In2S3} and \ce{In2S3}:\ce{Eu^{3+}} Nanoparticles}}
	\end{itemize}
	\vfill
}

\frame{
	\frametitle{}
	\vfill
	\begin{itemize}
		\item Jednostěnné nanotrubice \ce{In2S3} lze připravit difůzí z rozpouštědla (oleylamin) do kapaliny, kde jsou nerozpustné (ethanol).\footnote[frame]{\href{https://doi.org/10.1002/prac.18630890156}{Vorläufige Notiz über ein neues Metal}}
	\end{itemize}
	\begin{figure}
		\adjincludegraphics[width=0.75\textwidth]{img/Indium(III)_sulfide_nanotubes.jpg}
		\caption*{\ce{In2S3} nanotrubice.\footnote[frame]{Zdroj: \href{https://commons.wikimedia.org/wiki/File:Indium(III)_sulfide_nanotubes.jpg}{Bing Ni et al./Commons}}}
	\end{figure}
	\vfill
}

\frame{
	\frametitle{}
	\vfill
	\begin{itemize}
		\item \textit{Selenid inditý}, \ce{In2Se3}, je perspektivní fotovoltaický materiál.
		\item Lze jej připravit pomocí MOCVD směsi: trimethylindia, selanu a vodíku.\footnote[frame]{\href{https://doi.org/10.1063/1.2382742}{Growth of single-phase \ce{In2Se3} by using metal organic chemical vapor deposition with dual-source precursors}}
		\item \ce{2 (CH3)3In + 3 SeH2 ->[H2] In2Se3 + 6 CH4}
		\item \textit{Selenid indnatý}, \ce{InSe}, je polovodič s vrstevnatou strukturou.\footnote[frame]{\href{https://doi.org/10.1038/s41598-017-03186-x}{Indium selenide: an insight into electronic band structure and surface excitations}}
		\item Vrstvy se skládají z motivu \ce{Se-In-In-Se}.
		\item \textit{Tellurid inditý}, \ce{In2Te3}, je modrá pevná látka, která se působením silných kyselin rozkládá za vzniku tellanu.
	\end{itemize}
	\vfill
}

\frame{
	\frametitle{}
	\vfill
	\begin{figure}
		\begin{figure}
			\adjincludegraphics[height=0.7\textheight]{img/Indium_selenide_polytopes.png}
			\caption*{Modifikace \ce{InSe}.\footnote[frame]{Zdroj: \href{https://commons.wikimedia.org/wiki/File:Indium_(II)_selenide_polytopes.png}{Boukhvalov, D.W. et al./Commons}}}
		\end{figure}
	\end{figure}
	\vfill
}

\frame{
	\frametitle{}
	\vfill
	\begin{itemize}
		\item \textit{Sulfid thallný}, \ce{Tl2S}, je černá krystalická látka.
		\item Má fotovodivé vlastnosti, proto se využíval při konstrukci tzv. \textit{Thalofidových článků}\footnote[frame]{\href{https://doi.org/10.1103/PhysRev.15.289}{"Thalofide Cell"—a New Photo-Electric Substance}} pro detekci infračerveného záření.\footnote[frame]{\href{https://doi.org/10.2478/s11772-012-0037-7}{History of infrared detectors}}
		\item V přírodě se vyskytuje jako minerál karlinit.\footnote[frame]{\href{http://webmineral.com/data/Carlinite.shtml}{Carlinite Mineral Data}}
		\item \textit{Tellurid thallný}, \ce{Tl2Te}, byl zatím charakterizován pouze pomocí RTG strukturní analýzy.\footnote[frame]{\href{https://doi.org/10.1107/S0108270102005085}{\ce{Tl2Te} and its relationship with \ce{Tl5Te3}}}
		\item Byl připraven sléváním kovů při teplotě 280~$^\circ$C po dobu 160 hodin v~evakuované křemenné trubici.
	\end{itemize}
	\vfill
}

\subsection{Halogenidy}
\subsubsection{Halogenidy -- oxidační číslo III}
\frame{
	\frametitle{}
	\vfill
	\begin{itemize}
		\item Trifluoridy jsou netěkavé a mají vyšší teploty tání než těžší trihalogenidy.
		\item \ce{GaF3} je bílá pevná látka, taje nad teplotou 1000~$^\circ$C.
		\item Ostatní halogenidy gallité mají dimerní strukturu s podstatně nižšími teplotami tání.
		\item Připravují se pomocí MOCVD.
		\item Tvoří adukty typu \ce{MX3L}, \ce{MX3L2} a \ce{MX3L3}.
	\end{itemize}
	\begin{columns}
		\begin{column}{.5\textwidth}
			\begin{tabular}{|c|c|}
				\hline
				& Teplota tání [$^\circ$C] \\\hline
				\ce{GaCl3} & 78 \\\hline
				\ce{GaBr3} & 122 \\\hline
				\ce{GaI3} & 212 \\\hline
			\end{tabular}
		\end{column}
		\begin{column}{.5\textwidth}
			\adjincludegraphics[width=0.9\textwidth]{img/GaCl3.png}
		\end{column}
	\end{columns}
	\vfill
}

\frame{
\frametitle{}
\vfill
\begin{itemize}
	\item Fluorid gallitý lze připravit reakcí fluoru nebo fluorovodíku s oxidem gallitým nebo tepelným rozkladem \ce{(NH4)3GaF6}.
	\item Chlorid gallitý lze připravit přímou reakcí prvků nebo zahříváním oxidu s thionylchloridem:
	\item \ce{2 Ga + 3 Cl2 -> 2 GaCl3}
	\item \ce{Ga2O3 + 3 SOCl2 -> 2 GaCl3 + 3 SO2}
	\item Je to Lewisovská kyselina, čehož se využívá při katalýze organických reakcí, např. Friedel-Craftsových.
	\item Bromid i jodid gallitý můžeme také připravit přímým slučováním prvků.
	\item Bromid gallitý se podobně jako chlorid využívá jako katalyzátor v organické syntéze.
	\item Jodid gallitý lze redukovat galliem na jodid gallný.
	\item \ce{GaI3 + 2 Ga -> 3 GaI}
\end{itemize}
\vfill
}

\frame{
	\frametitle{}
	\vfill
	\begin{itemize}
		\item Fluorid inditý lze získat reakcí oxidu inditého s fluorovodíkem nebo kyselinou fluorovodíkovou. Struktura se skládá z oktaedrů \ce{InF6} propojených vrcholy.
		\item Chlorid, bromid i jodid mají dimerní strukturu podobnou \ce{Al2Cl6}.
		\item Chlorid inditý lze podobně jako gallitý, připravit slučováním prvků.
		\begin{itemize}
			\item Tvoří několik řad aduktů a jeho Lewisovské kyselosti se využívá v katalýze, např. při Diels-Alderových reakcích.\footnote[frame]{\href{https://doi.org/10.1016/S0040-4020(02)00907-9}{\ce{InCl3}-Catalyzed hetero-Diels–Alder reaction}}
			\item Slouží jako výchozí látka k produkci organokovových sloučenin india.
			\item \ce{InCl3 + 3 EtMgBr ->[Et2O] Et3In.OEt2 + 3 MgBr2}
			\item Za vyšších teplot reaguje s kovovým indiem za vzniku nižších halogenidů: \ce{In5Cl9}, \ce{In2Cl3} a \ce{InCl}.\footnote[frame]{\href{https://doi.org/10.1107/S0365110X66002032}{The crystal structure of the room temperature modification of indium chloride, InCl}}
		\end{itemize}
		\item Bromid inditý lze také připravit z prvků, využívá se jako katalyzátor ve Friedel-Craftsových reakcích.\footnote[frame]{\href{https://doi.org/10.1021/jo9014613}{\ce{InBr3}: A Versatile Catalyst for the Different Types of Friedel-Crafts Reactions}}
		\item Jodid inditý získáme reakcí india s parami jodu, vodné roztoky lze připravit reakcí s kyselinou jodovodíkovou.
	\end{itemize}
	\vfill
}

\frame{
	\frametitle{}
	\vfill
	\begin{itemize}
		\item \textit{Fluorid thallitý} je bílá krystalická látka, taje při 550~$^\circ$C, thallium má koordinační číslo 9, krystalová struktura odpovídá \ce{YF3}.
		\item Připravit lze fluorací oxidu pomocí fluoru, \ce{BrF3} nebo \ce{SF4} za zvýšené teploty.
		\item \textit{Chlorid thallitý} má strukturu podobnou \ce{AlCl3}, je nestabilní, při 40~$^\circ$C se rozkládá na \ce{TlCl}.
		\item Lze jej připravit reakcí \ce{TlCl} s plynným chlorem v acetonitrilu.
		\item \textit{Bromid thallitý} se rozládá už při teplotách pod 40~$^\circ$C.
		\item Lze jej připravit reakcí \ce{TlCl} s bromem v acetonitrilu nebo ve vodném roztoku reakcí bromidu s \ce{TlBr}.
		\item \textit{Jodid thallitý} není znám. Sloučenina se vzorcem \ce{TlI3} je známa, ale jde o \textit{trijodid thallný}, tzn. obsahuje lineární anion \ce{I3^-}.
		\item Lze jej připravit reakcí jodidu thallného s jodem v prostředí kyseliny jodovodíkové.
		\item \ce{TlI + I2 ->[HI] TlI3}
	\end{itemize}
	\begin{center}
		\adjincludegraphics[width=0.3\textwidth]{img/Triiodide.png}
	\end{center}
	\vfill
}

\subsubsection{Monohalogenidy}
\frame{
	\frametitle{}
	\vfill
	\begin{itemize}
		\item \textit{Monohalogenidy} vytvářejí všechny čtyři kovy 13. skupiny. U hliníku se jedná o nestabilní částice existující pouze jako dvouatomové molekuly s velmi krátkým časem života nebo za kryogenních podmínek. Jejich stabilita vzrůstá s protonovým číslem prvku.
		\item GaF a InF jsou nestabilní a existují pouze v plynném stavu. Další halogenidy jsou stabilnější, lze je připravit redukcí trojmocných halogenidů kovovým prvkem.
		\item \ce{GaCl3 + 2 Ga -> 3 GaCl}
		\item Podobnou reakcí lze získat i komplexní halogenidy \ce{Ga[GaCl4]}, \ce{Ga[GaBr4]} a \ce{Ga[GaI4]}.
		\item Zahříváním equimolární směsi halogenidu gallitého s kovovým galliem nebo halogenací gallia pomocí \ce{Hg2X2} nebo \ce{HgX2} lze připravit komplexy \ce{Ga^I[Ga^{III}X4]}.
	\end{itemize}
	\vfill
}

\frame{
	\frametitle{}
	\vfill
	\begin{tabular}{|c|c|c|c|r@{,}l|}
		\hline
		Látka & Barva & T. tání [$^\circ$C] & T. varu [$^\circ$C] &
		\multicolumn{2}{|c|}{Rozp. [g/100 g \ce{H2O}]} \\\hline
		TlF & bílý & 322 & 826 & 80 & 0 \\\hline
		TlCl & bílý & 431 & 720 & 0 & 33 \\\hline
		TlBr & světle žlutý & 460 & 815 & 0 & 058 \\\hline
		TlI & žlutý & 442 & 823 & 0 & 006 \\\hline
	\end{tabular}

	\begin{itemize}
		\item Halogenidy thallné jsou stabilními halogenidy.
		\item TlF lze připravit reakcí kyseliny fluorovodíkové s \ce{Tl2CO3}. Je dobře rozpustný ve vodě, má defektní strukturu NaCl.
		\item \ce{Tl2CO3 + 2 HF -> 2 TlF + CO2 + H2O}
		\item Ostatní halogenidy thallné připravíme srážením okyselených roztoků thallných solí příslušným halogenidem.
	\end{itemize}
	\begin{align*}
		\ce{TlClO4 + KCl &-> TlCl + KClO4}\\
		\ce{TlClO4 + KBr &-> TlBr + KClO4}\\
		\ce{Tl2SO4 + 2 KI &-> 2 TlI + K2SO4}
	\end{align*}
	\vfill
}

\subsubsection{KRS - směsné halogenidy thallné}
\frame{
	\frametitle{}
	\vfill
	\begin{itemize}
		\item Směsné halogenidy thallné slouží jako materiály pro optiku infračervených spektrometrů.\footnote[frame]{\href{https://www.chemeurope.com/en/encyclopedia/Thallium_halides.html}{Thallium halides}}
		\item Označují se jako KRS (Kristalle aus dem Schmelz-fluss -- krystaly z taveniny).
		\item Nejpoužívanější jsou KRS-5 (červený, \ce{TlBr_{0.4}I_{0.6}}) a KRS-6 (bezbarvý, \ce{TlBr_{0.3}Cl_{0.7}}).
		\item Poprvé byly připraveny v roce 1941 v laboratoři Carl Zeiss Optical Works v Jeně.
		\item Jejich výhodou oproti běžně používanému KBr je nízká rozpustnost ve vodě.
	\end{itemize}
	\begin{center}
		\adjincludegraphics[width=0.5\textwidth]{img/KRS-okna.jpg}
	\end{center}
	\vfill
}

\subsubsection{Další halogenidy}
\frame{
	\frametitle{}
	\begin{columns}
		\begin{column}{0.65\textwidth}
			\vfill
			\begin{itemize}
				\item Gallium vytváří řadu dvojjaderných komplexních sloučenin obsahujících centrální ion \ce{Ga$_2^{2+}$} s délkou vazby \ce{Ga-Ga} 240,6~pm.
				\item Sumární vzorec těchto komplexů je \ce{[Ga2X2L2]}, např. komplex s 1,4-dioxanem \ce{[Ga2Cl2(C4H8O8)2]}.
				\item Koordinační okolí Ga je téměř tetraedrické a sloučenina má zákrytovou konformaci.
				\item Známe také řadu stabilní dihalogenidů gallia, které lze připravit zahříváním \ce{GaX3} s galliem nebo halogenací gallia pomocí \ce{HgX2} a \ce{H2X2}.
				\item Jde např. o: \ce{Ga^I[Ga^{III}Cl4]}, \ce{Ga[GaBr4]} nebo \ce{[Ga^IL4][Ga^{III}X4]}
			\end{itemize}
			\vfill
		\end{column}
		\begin{column}{0.4\textwidth}
			\adjincludegraphics[height=0.6\textheight]{img/Ga2Cl4Diox2.png}
		\end{column}
	\end{columns}
}

\subsection{Nitridy}
\frame{
	\frametitle{}
	\begin{columns}
		\begin{column}{0.65\textwidth}
			\vfill
			\begin{itemize}
				\item \textit{Nitrid gallitý}, GaN, je polovodivý materiál využívaný LED diodách, např. v LASERech pro Blue-ray mechaniky. Dopováním můžeme získat barvy v rozmezí od červené po ultrafialovou.
				\item Vyrábí se reakcí gallia nebo oxidu gallitého s amoniakem za vysoké teploty.
				\item \ce{2 Ga + 2 NH3 ->[1050 $^\circ$C] 2 GaN + 3 H2}
				\item \ce{Ga2O3 + 2 NH3 ->[1050 $^\circ$C] 2 GaN + 3 H2O}
				\item \textit{Nitrid inditý}, InN, je také polovodivý.
				\item Společně s GaN vytváří ternární systém \textit{InGaN} se šířkou zakázaného pásu v rozmezí IR až UV části spektra. Proto je velmi zajímavým materiálem pro konstrukci solárních článků.
			\end{itemize}
			\vfill
		\end{column}
		\begin{column}{0.4\textwidth}
			\begin{figure}
				\adjincludegraphics[width=\textwidth]{img/Crystal-GaN.jpg}
				\caption*{Krystal \ce{GaN}.\footnote[frame]{Zdroj: \href{https://commons.wikimedia.org/wiki/File:Crystal-GaN.jpg}{Opto-p/Commons}}}
			\end{figure}
			\vfill
		\end{column}
	\end{columns}
}

\frame{
	\frametitle{}
			\vfill
			\begin{itemize}
				\item \textit{Nitrid thallitý}, \ce{TlN}, nebyl dlouho znám. Připraven byl až reakcí thallia s dusíkem v obloukovém výboji.\footnote[frame]{\href{https://doi.org/10.1016/j.matchemphys.2019.02.005}{Synthesis and properties of thallium nitride films}}
				\item Krystaluje ve stejné soustavě jako wurtzit (\ce{(Zn,Fe)S}).
				\item Snadno se na vzduchu oxiduje za vzniku \ce{Tl2O3}.
				\item \textit{Nitrid thallný}, \ce{Tl3N}, je černá, pevná látka vznikající srážením dusičnanu thallného amidem draselným v kapalném amoniaku.
				\item \ce{3 TlNO3 + 3 KNH2 ->[NH3(l)] Tl3N + 3 KNO3 + 2 NH3}
			\end{itemize}
			\begin{figure}
				\adjincludegraphics[height=.25\textheight]{img/Wurtzite.png}
				\caption*{Základní buňka wurtzitu.\footnote[frame]{Zdroj: \href{https://commons.wikimedia.org/wiki/File:Wurtzite.png}{Kent G. Budge/Commons}}}
			\end{figure}
			\vfill
}

\frame{
	\frametitle{}
	\vfill
	\begin{itemize}
		\item \textit{InGaN, nitrid indito-gallitý}, \ce{In_xGa_{1-x}N}, je polovodivý materiál vyráběný jako směs nitridu gallitého a inditého.
		\item Využívá se ke konstrukci modrých LED a fotovoltaických článků.\footnote[frame]{\href{https://doi.org/10.1016/S0022-0248(98)01344-X}{InGaN-based blue light-emitting diodes and laser diodes}}
	\end{itemize}
	\begin{columns}
		\begin{column}{.5\textwidth}
			\begin{figure}
				\adjincludegraphics[width=.9\textwidth]{img/White_LED.png}
				\caption*{Spektrum bílé LED.\footnote[frame]{Zdroj: \href{https://commons.wikimedia.org/wiki/File:White_LED.png}{Deglr6328/Commons}}}
			\end{figure}
		\end{column}
		\begin{column}{.5\textwidth}
			\begin{figure}
				\adjincludegraphics[width=.9\textwidth]{img/Uv-LED.jpg}
				\caption*{UV LED.\footnote[frame]{Zdroj: \href{https://commons.wikimedia.org/wiki/File:Uv-LED.jpg}{C. Pelant /Commons}}}
			\end{figure}
		\end{column}
	\end{columns}
	\vfill
}

\subsection{Organokovové sloučeniny}
\frame{
	\frametitle{}
	\vfill
	\begin{itemize}
		\item Chemická reaktivita vazby \ce{M-C} klesá v pořadí Al$>$Ga$\approx$In$>$Tl.
		\item Sloučeniny \ce{GaR3} se připravují alkylací gallia pomocí \ce{HgR2} nebo pomocí Grignardových činidel (\ce{RMgBr}), příp. reakcí \ce{AlR3} a \ce{GaCl3}. Jsou to reaktivní, pohyblivé kapaliny.
		\item \ce{2 Ga + 3 Me2Hg -> 2 Me3Ga + 3 Hg}
		\item \ce{GaCl3 + 3 MeMgBr -> Me3Ga + 3 MgBrCl}
		\item Trimethylgallan je prekurzorem gallia pro MOVPE (MetalOrganic Vapour Phase Epitaxy) přípravu polovodičů, např. GaAs nebo GaN pro LED.\footnote[frame]{\href{https://doi.org/10.1002/9780470132623.ch8}{Trimethylindium and Trimethylgallium}}
		\item \ce{Ga(CH3)3 + AsH3 ->[1300 $^\circ$C] GaAs + CH4}
		\item Trifenylové sloučeniny gallia jsou v roztoku monomerní, ale v pevném stavu vytváří agregáty pomocí interakcí \ce{M\dotsm C}.
	\end{itemize}
	\begin{center}
		\adjincludegraphics[width=0.4\textwidth]{img/Trimethylgallium-2D.png}
	\end{center}
	\vfill
}

\frame{
	\frametitle{}
		\vfill
	\begin{columns}
	\begin{column}{0.65\textwidth}
		\begin{itemize}
		\item Sloučeniny v oxidačním čísle I jsou u india běžnější než u gallia.
		\item Příkladem může být \textit{cyklopentadienylindium}, které se připravuje reakcí chloridu s cyklopentadienyllithiem.\footnote[frame]{\href{https://doi.org/10.1039/DT9810002592}{A simple synthesis of cyclopentadienylindium(I)}}
		\item \ce{InCl + CpLi ->[Et2O] CpIn + LiCl}
		\item V pevném stavu vytváří cik-cak řetězce, ve kterých se střídají ionty indné a cyklopentadienidové kruhy.
		\end{itemize}
		\vfill
		\end{column}
		\begin{column}{0.4\textwidth}
		\adjincludegraphics[width=0.7\textwidth]{img/Cyclopentadienylindium.png}
		\end{column}
		\end{columns}
	\vfill
}

\frame{
	\frametitle{}
	\vfill
	\begin{figure}
		\adjincludegraphics[width=\textwidth]{img/InCp-chain-3D-balls.png}
		\caption*{Řetězec cyklopentadienylindia.\footnote[frame]{Zdroj: \href{https://commons.wikimedia.org/wiki/File:InCp-chain-3D-balls.png}{Ben Mills/Commons}}}
	\end{figure}
	\vfill
}

\frame{
	\frametitle{}
	\vfill
	\begin{columns}
		\begin{column}{.65\textwidth}
			\begin{itemize}
				\item V oxidačním čísle III známe více sloučenin.
				\item Nejdůležitější sloučeninou je \textit{trimethylindium} (\ce{InMe3}), na rozdíl od trimethylhliníku má monomerní strukturu.
				\item Připravuje se methylací chloridu inditého:\footnote[frame]{\href{https://doi.org/10.1002/9780470132623.ch8}{Trimethylindium and Trimethylgallium}}
				\item \ce{InCl3 + 3 LiMe ->[Et2O] Me3In.OEt2 + 3 LiCl}
				\item Vytváří bezbarvé, jehličkovité krystaly, které jsou na vzduchu pyroforické. Tají při 88~$^\circ$C.
				\item Využívá se pro MOVPE syntézu polovodičových materiálů -- InP, InN, GaInAs, AlInGaNP, $\dots$
			\end{itemize}
		\end{column}

		\begin{column}{.4\textwidth}
			\begin{figure}
				\adjincludegraphics[width=\textwidth]{img/Trimethylindium-from-xtal-3D-balls.png}
				\caption*{Struktura trimethylindia.\footnote[frame]{Zdroj: \href{https://commons.wikimedia.org/wiki/File:Trimethylindium-from-xtal-3D-balls.png}{Ben Mills/Commons}}}
			\end{figure}
		\end{column}
	\end{columns}
	\vfill
}

\frame{
	\frametitle{}
	\vfill
	\begin{itemize}
		\item \textbf{Indium-mediated allylations} (Allylace zprostředkovaná indiem)\footnote[frame]{\href{https://doi.org/10.1021/acs.accounts.6b00362}{Indium-Mediated Stereoselective Allylation}}
		\item Reakci lze rozdělit na dva kroky, nejprve dochází k reakci kovového india s allylhalidem za vzniku organokovového meziproduktu RInLX.
		\item Poté dochází k adici na karbonyl.
	\end{itemize}
	\begin{center}
		\adjincludegraphics[width=0.8\textwidth]{img/IMA_in_two_steps_as_scalable_vg_file.png}
	\end{center}
	\vfill
}

\frame{
	\frametitle{}
	\vfill
	\begin{itemize}
		\item Thallium, stejně jako indium, běžně vytváří sloučeniny v oxidačním čísle I, např. cyklopentadienylthallium (TlCp), které má podobnou strukturu jako indný analog.
		\item Vyrábí se ze síranu thallného a cyklopentadienu.\footnote[frame]{\href{https://doi.org/10.1002/9780470132555.ch31}{Cyclopentadienylthallium (Thallium Cyclopentadienide)}}
		\item \ce{Tl2SO4 + 2 NaOH -> 2 TlOH + Na2SO4}
		\item \ce{TlOH + C5H6 -> TlC5H5 + H2O}
		\item TlCp dokáže přenášet cyklopentadienyl na kovy z d-bloku.\footnote[frame]{\href{https://doi.org/10.1016/j.jorganchem.2017.11.003}{Indenyl rhodium complexes. Synthesis and catalytic activity in reductive amination using carbon monoxide as a reducing agent}}
		\item Ale dokáže vystupovat i jako akceptor cyklopentadiendového aniontu, např. reakcí s \ce{MgCp2} vzniká aniont \ce{[TlCp2]^-}.
	\end{itemize}
	\vfill
}

\frame{
	\frametitle{}
	\vfill
	\begin{figure}
		\adjincludegraphics[width=.8\textwidth]{img/TlCpPolymer.png}
		\caption*{Struktura TlCp.\footnote[frame]{Zdroj: \href{https://commons.wikimedia.org/wiki/File:TlCpPolymer.png}{Smokefoot/Commons}}}

		\adjincludegraphics[width=.8\textwidth]{img/Cyclopentadienylthallium-chain-from-xtal-3D-sf.png}
		\caption*{Struktura TlCp.\footnote[frame]{Zdroj: \href{https://commons.wikimedia.org/wiki/File:Cyclopentadienylthallium-chain-from-xtal-3D-sf.png}{Ben Mills/Commons}}}
	\end{figure}
	\vfill
}

\frame{
	\frametitle{}
	\vfill
	\begin{itemize}
		\item Trifluoroctan thallitý se připravuje dvanáctihodinovým refluxem oxidu thallitého v kyselině trifluoroctové.\footnote[frame]{\href{https://doi.org/10.1021/jo5011087}{Ketene–Ketene Interconversion. 6-Carbonylcyclohexa-2,4-dienone–Hepta-1,2,4,6-tetraene-1,7-dione–6-Oxocyclohexa-2,4-dienylidene and Wolff Rearrangement to Fulven-6-one}}
		\item Lze jej využít k elektrofilní thalliaci aromatických sloučenin, příkladem může být jodace \textit{p}-xylenu.\footnote[frame]{\href{https://doi.org/10.15227/orgsyn.055.0070}{2-iodo-p-xylene}}
		\item Během reakce dochází ke vzniku vazby \ce{Tl-C}.
	\end{itemize}
	\begin{figure}
		\adjincludegraphics[width=\textwidth]{img/Arene_thallation.png}
		\caption*{Jodace p-xylenu.\footnote[frame]{Zdroj: \href{https://commons.wikimedia.org/wiki/File:Arene_thallation.svg}{V8rik/Commons}}}
	\end{figure}
	\vfill
}

\input{../Last}

\end{document}